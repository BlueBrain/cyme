\documentclass{beamer}
\mode<presentation>
\usepackage{tikz}
\usetikzlibrary{lindenmayersystems}
\usetikzlibrary{automata,arrows}
\usetikzlibrary{petri}
\usetikzlibrary[shadings]
\usetikzlibrary{arrows}
\usetikzlibrary{trees,snakes}

\usepackage{graphicx}
\usepackage{graphics}
\usepackage{amsfonts}
\usepackage{epstopdf}
\usepackage{pifont}
\usepackage{color}
\usepackage{amsmath}
\usepackage{psfrag}
\usepackage{alltt}
\usepackage{color,listings}

\usepackage{xcolor,colortbl}
\usepackage{color}
\usepackage{epsfig}

\graphicspath{{plot/}{figure/}{logo/}}

%\usefonttheme{professionalfonts}
%\usepackage{beamerthemesplit}

%MAdrid
%\usetheme[secheader]{Madrid}
\usetheme[compress]{Berlin}
%\usetheme{Berkeley}
%\usetheme[compress]{Ilmenau}
\usecolortheme{seahorse}
%\usecolortheme{default}
%\usecolortheme{sidebartab}
%\usecolortheme{albatross}
%\usecolortheme{beetle} 
%\usecolortheme{crane}
%\usecolortheme{dove} 
%\usecolortheme{fly} 
%\usecolortheme{seagull}
%\usecolortheme{lily}
%\usecolortheme{orchid}
%crane
%\beamertemplatetransparentcoverdhigh
%\useoutertheme[footline=author]{miniframes}
%\usefonttheme[onlylarge]{structuresmallcapsserif}
%\setbeamerfont{title}{shape=\itshape,family=\rmfamily}
\usepackage[frenchb]{babel}
\usepackage[applemac]{inputenc}

          \definecolor{C0}{RGB}{207,46,49}
          \definecolor{C1}{RGB}{226,162,45}
          \definecolor{C2}{RGB}{227,189,28}
          \definecolor{C3}{RGB}{162,179,36}
          \definecolor{C4}{RGB}{29,154,47}


\newcommand\cad{
\tikzstyle{level 1}=[sibling angle=120]
\tikzstyle{level 2}=[sibling angle=60]
\tikzstyle{level 3}=[sibling angle=30]
\tikzstyle{every node}=[fill]
\tikzstyle{edge from parent}=[snake=expanding waves,segment length=1mm,segment angle=10,draw]
\begin{tikzpicture}[grow cyclic,shape=circle,very thick,level distance=13mm,   cap=round, rotate=90,scale=0.7]
\node {} child [color=\A] foreach \A in {C0}
    { node {} child [color=\A!50!\B] foreach \B in {C2,C3,C4}
        {  node {}        }
    };
\end{tikzpicture}}

%\title[TMAC in Microtube]{Tangential Momemtum Accomodation in Microtube}
\title{Cyme framework}
\author{Tim Ewart and Fabien Delalondre}
%\subtitle{\texttt{timothee.ewart@polytech.univ-mrs.fr}
 \institute[HPC team, BBP]{

\tikzstyle{level 1}=[sibling angle=120]
\tikzstyle{level 2}=[sibling angle=60]
\tikzstyle{level 3}=[sibling angle=30]
\tikzstyle{every node}=[fill]
\tikzstyle{edge from parent}=[snake=expanding waves,segment length=1mm,segment angle=10,draw]
\begin{tikzpicture}[grow cyclic,shape=circle,very thick,level distance=13mm,   cap=round, rotate=90,scale=0.7]
\node {} child [color=\A] foreach \A in {C0}
    { node {} child [color=\A!50!\B] foreach \B in {C2,C3,C4}
        { node {} child [color=\A!50!\B!50!\C] foreach \C in {C2,C3,C4}
            { node {} }
        }
    };
\end{tikzpicture}
 }



\date[09/03/13]{\includegraphics[scale=0.3]{EPFL_LOG_QUADRI_Red.eps}   }



\begin{document}





\setbeamercolor{structure}{fg=C2, bg=C1}

\lstset{language=C++,
   keywords={break,case,catch,continue,else,elseif,end,for,function,const,begin,double,float
      global,if,otherwise,persistent,return,switch,try,while,constexpr,static,inline,size_t,typename
     mulsd,addsd,mulsd,divsd,movl},
   keywords=[2]{class,struct,template},
   keywords=[3]{movapd,AoS,AoSoA},
   basicstyle=\ttfamily,
   keywordstyle=\color{C0},
   keywordstyle=[2]\color{C1},
   keywordstyle=[3]\color{C2},
   commentstyle=\color{C3},
   stringstyle=\color{dkgreen},
   numbers=left,
   numberstyle=\tiny\color{gray},
   stepnumber=1,
   numbersep=10pt,
   backgroundcolor=\color{white},
   tabsize=4,
   showspaces=false,
   frame=single,
   showstringspaces=false}


\begin{frame}
\titlepage
\end{frame}


%%%%%%%%%%%%%%%%%%%%%%%%%% EXP 0
\section[Structure of data]{which structure}
\subsection*{What is the Neurodamus issue ?}
\begin{frame}
Neurodamus is a synapse problem
\begin{enumerate}
\item 10 000 synapses per neurons 
\item Typed Synapses (chemical processes )
\item Synapses represented by (5-20 numbers) 
\end{enumerate}
\center
How do we maximize the  calculation ?
\end{frame}

%%%%%%%%%%%%%%%%%%%%%%%%%% EXP 1
\subsection*{Which data layout}
\begin{frame}
Problem: 
\begin{enumerate}
\item How do we represent a series of small objects ($e.g.$ 5 floats) ? 
\item How do we manipulate them ? 
\end{enumerate}
Solutions:
\begin{itemize}
\item Array of Structure  (AoS)
\item Structure of Array (SoA - interleave data)
\item  Array of Structure of Array (AoSoA - interleave data)
\item Serial or vectorial calculation (SIMD)
\end{itemize}

Cool:
\begin{itemize}
\item Run vectorial calculation only
\end{itemize}
 
\end{frame}



%%%%%%%%%%%%%%%%%%%%%%%%%% EXP 2

\subsection*{AoS - SoA - AoSoA, what is it ?}
\begin{frame}[fragile]
An object (5 elements), and data layout 

\begin{tikzpicture}[scale=1.4]
          \path[fill=C0] (0,0) rectangle ++(0.2,0.2); 
          \path[fill=C1] (0.2,0) rectangle ++(0.2,0.2); 
          \path[fill=C2] (0.4,0) rectangle ++(0.2,0.2); 
          \path[fill=C3] (0.6,0) rectangle ++(0.2,0.2); 
          \path[fill=C4] (0.8,0) rectangle ++(0.2,0.2);
          \draw[step=.2,help lines] (0,0) grid (1,0.2); 
\end{tikzpicture}

\hrulefill


AoS 

\begin{tikzpicture}[scale=1.4]
      \foreach \x in {0.0,1,...,4} {
          \path[fill=C0] (\x,0) rectangle ++(0.2,0.2); 
          \path[fill=C1] (\x+0.2,0) rectangle ++(0.2,0.2); 
          \path[fill=C2] (\x+0.4,0) rectangle ++(0.2,0.2); 
          \path[fill=C3] (\x+0.6,0) rectangle ++(0.2,0.2); 
          \path[fill=C4] (\x+0.8,0) rectangle ++(0.2,0.2);
      }   
\draw[step=.2,help lines] (0,0) grid (5,0.2); 
\end{tikzpicture}

\hrulefill


SoA following 256 bit vector register (avx - qpx)


\begin{tikzpicture}[scale=1.4]
       \path[fill=C0] (0,0) rectangle ++(1,0.2); 
       \draw [>=stealth,<->] (0,-0.1)--(0.8,-0.1) ;
       \draw [>=stealth,<->] (0.8,-0.1)--(1.6,-0.1) ;
       \draw (0.4,-0.1) node[below]{avx} ;
       \draw (1.2,-0.1) node[below]{avx} ;

       \path[fill=C1] (1.6,0) rectangle ++(1,0.2); 
       \path[fill=C2] (3.2,0) rectangle ++(1,0.2); 
       \path[fill=C3] (4.8,0) rectangle ++(1,0.2); 
      \path[fill=C4] (6.4,0) rectangle ++(1,0.2);    
\draw[xstep=.8,ystep=.2,help lines] (0,0) grid (8,0.2); 
\end{tikzpicture}

\vspace{0.5cm}

AoSoA following  256 bit vector register (avx - qpx)  

\begin{tikzpicture}[scale=1.4]
       \path[fill=C0] (0,0) rectangle ++(0.8,0.2); 
       \draw [>=stealth,<->] (0,-0.1)--(0.8,-0.1) ;
       \draw [>=stealth,<->] (0.8,-0.1)--(1.6,-0.1) ;
       \draw (0.4,-0.1) node[below]{avx} ;
       \draw (1.2,-0.1) node[below]{avx} ;

       \path[fill=C1] (0.8,0) rectangle ++(0.8,0.2); 
       \path[fill=C2] (1.6,0) rectangle ++(0.8,0.2); 
       \path[fill=C3] (2.4,0) rectangle ++(0.8,0.2); 
       \path[fill=C4] (3.2,0) rectangle ++(0.8,0.2);  
       \path[fill=C0] (4,0) rectangle ++(0.2,0.2); 
       \path[fill=C1] (4.8,0) rectangle ++(0.2,0.2); 
       \path[fill=C2] (5.6,0) rectangle ++(0.2,0.2); 
       \path[fill=C3] (6.4,0) rectangle ++(0.2,0.2); 
       \path[fill=C4] (7.2,0) rectangle ++(0.2,0.2); 
\draw[xstep=.8,ystep=.2,help lines] (0,0) grid (8,0.2); 
\end{tikzpicture}
\begin{center}
which layout ? Which container ?
\end{center}
\end{frame}



%%%%%%%%%%%%%%%%%%%%%%%%%% EXP 3
%\subsection*{Serial or vectorial calculation,  what is it ?}
%\begin{frame}[fragile]
%
%Processors  calculation can be serial and vectorial:  
%
%\begin{minipage}{0.4\textwidth}
%\centering
%\vspace{0.5cm}
%Serial \\
%\vspace{0.5cm}
%\begin{tikzpicture}[scale=1.4]
%         \path[fill=C0] (0,0) rectangle ++(0.2,0.2); 
%         \draw[step=.2,help lines] (0,0) grid (0.2,0.2); 
%         \path[fill=C0] (0,-0.8) rectangle ++(0.2,0.2); 
%         \draw (0.1,-0.1) node[below]{+=} ;
%        \draw[step=.2,help lines] (0,-0.8) grid (0.2,-0.6); 
%\end{tikzpicture}
%
%1 addition
%\end{minipage}
%\begin{minipage}{0.4\textwidth}
%\centering
%\vspace{0.5cm}
%Vectorial (AVX 4 double) \\
%\vspace{0.5cm}
%\begin{tikzpicture}[scale=1.4]
%       \path[fill=C0] (0,0) rectangle ++(0.8,0.2);     
%       \draw[step=.2,help lines] (0,0) grid (0.8,0.2);
%       \draw (0.4,-0.1) node[below]{+=} ;
%       \path[fill=C0] (0,-0.8) rectangle ++(0.8,0.2);     
%       \draw[step=.2,help lines] (0,-0.8) grid (0.8,-0.6);
%\end{tikzpicture}
%
%4 additions
%\end{minipage}
%
%\vspace{0.5cm}
%
%Vectorial calculation can give a nice speed up ($\times 4$). 
%
%Serial calculation are usuals for the compiler. It may generate vectorial instruction. Else by hand, It necessitates SoA or AoSoA data layout.
%
%
%\end{frame}

%%%%%%%%%%%%%%%%%%%%%%%%%% EXP 4

\subsection*{An universal container}
\begin{frame}[fragile]
Static container with two data layout
 \begin{lstlisting}[flexiblecolumns=true,basicstyle=\sffamily]     
      block<double,5,5,AoSoA> block_AoS;       
      block<double,5,5,AoS> block_AoSoA;
\end{lstlisting}

\hspace{1cm}

Memory layout:

\vspace{0.5cm}

\begin{tikzpicture}[scale=1.2]

\vspace{0.5cm}

              \draw (-0.35,0.35) node[below]{AoS} ;
              \draw (-0.5,-0.65) node[below]{AoSoA} ;

       \foreach \x in {0.0,1,...,4} {
          \path[fill=C0] (\x,0) rectangle ++(0.2,0.2); 
          \path[fill=C1] (\x+0.2,0) rectangle ++(0.2,0.2); 
          \path[fill=C2] (\x+0.4,0) rectangle ++(0.2,0.2); 
          \path[fill=C3] (\x+0.6,0) rectangle ++(0.2,0.2); 
          \path[fill=C4] (\x+0.8,0) rectangle ++(0.2,0.2);
      }  
      
      
      \draw [>=stealth,->] (0.1,-0.1)--(0.1,-0.7) ;      
      \draw [>=stealth,->] (1.1,-0.1)--(0.3,-0.7) ;      
      \draw [>=stealth,->] (2.1,-0.1)--(0.5,-0.7) ;      
      \draw [>=stealth,->] (3.1,-0.1)--(0.7,-0.7) ;      
      \draw [>=stealth,->] (4.1,-0.1)--(4.1,-0.7) ;      

       AoSoA layout.
       \path[fill=C0] (0,-1) rectangle ++(0.8,0.2);              
       \path[fill=C1] (0.8,-1) rectangle ++(0.8,0.2); 
       \path[fill=C2] (1.6,-1) rectangle ++(0.8,0.2); 
       \path[fill=C3] (2.4,-1) rectangle ++(0.8,0.2); 
       \path[fill=C4] (3.2,-1) rectangle ++(0.8,0.2);  
       \path[fill=C0] (4,-1) rectangle ++(0.2,0.2); 
       \path[fill=C1] (4.8,-1) rectangle ++(0.2,0.2); 
       \path[fill=C2] (5.6,-1) rectangle ++(0.2,0.2); 
       \path[fill=C3] (6.4,-1) rectangle ++(0.2,0.2); 
       \path[fill=C4] (7.2,-1) rectangle ++(0.2,0.2); 
       \draw[xstep=.8,ystep=.2,help lines] (0,-1) grid (8,-0.8); 
       \draw[step=.2,help lines] (0,0) grid (5,0.2); 


       \draw [>=stealth,<->] (0,-1.1)--(0.8,-1.1) ;
       \draw (0.4,-1.1) node[below]{avx} ;

       \draw [>=stealth,<->] (4,-1.1)--(4.8,-1.1) ;
       \draw (4.4,-1.1) node[below]{avx} ;

       \draw [>=stealth,<->] (4.2,-0.7)--(4.8,-0.7) ;
       \draw (4.49,-0.25) node[below]{pad} ;
\end{tikzpicture}


\end{frame}

%%%%%%%%%%%%%%%%%%%%%%%%%% EXP 5

\subsection*{Generic syntax}
\begin{frame}[fragile]
\centering
Generic syntax: 
 \begin{lstlisting}[flexiblecolumns=true,basicstyle=\sffamily]     
      block<double,5,5,AoSoA/AoS> block;
      typename block<double,5,5,AoSoA/AoS>::iterator it;
      for(it = block.begin();it != block.end();++it){...}
\end{lstlisting}

\hspace{2cm}

Memory access:

\begin{tikzpicture}[scale=1.4]
       

       \foreach \x in {0.0,1,...,4} {
          \path[fill=C0] (\x,0) rectangle ++(0.2,0.2); 
          \path[fill=C1] (\x+0.2,0) rectangle ++(0.2,0.2); 
          \path[fill=C2] (\x+0.4,0) rectangle ++(0.2,0.2); 
          \path[fill=C3] (\x+0.6,0) rectangle ++(0.2,0.2); 
          \path[fill=C4] (\x+0.8,0) rectangle ++(0.2,0.2);
      }  
       \foreach \x in {0.0,1,...,3} {
            \draw (0.6+\x,0.8) node[below]{++it} ;
             \draw[->,>=latex] (0.1+\x,0.3) to[bend left] (1.1 + \x,0.3);
       }
            \draw[step=.2,help lines] (0,0) grid (5,0.2); 
            
            
       \path[fill=C0] (0,-1) rectangle ++(0.8,0.2);              
       \path[fill=C1] (0.8,-1) rectangle ++(0.8,0.2); 
       \path[fill=C2] (1.6,-1) rectangle ++(0.8,0.2); 
       \path[fill=C3] (2.4,-1) rectangle ++(0.8,0.2); 
       \path[fill=C4] (3.2,-1) rectangle ++(0.8,0.2);  
       \path[fill=C0] (4,-1) rectangle ++(0.2,0.2); 
       \path[fill=C1] (4.8,-1) rectangle ++(0.2,0.2); 
       \path[fill=C2] (5.6,-1) rectangle ++(0.2,0.2); 
       \path[fill=C3] (6.4,-1) rectangle ++(0.2,0.2); 
       \path[fill=C4] (7.2,-1) rectangle ++(0.2,0.2); 


        \draw (2,-0.1) node[below]{++it} ;
      \draw[->,>=latex] (0.1, -0.7) to[bend left=10] (4.1 ,-0.7);
 
       
       \draw[xstep=.8,ystep=.2,help lines] (0,-1) grid (8,-0.8); 
\end{tikzpicture}
\end{frame}

%%%%%%%%%%%%%%%%%%%%%%%%%% EXP 6
\section[expression-template]{expression-template}
\subsection*{Template-expression: expression-template 1}
\begin{frame}[fragile]
Evaluate: \\
\centering
$\boxed{a[0] = a[1]  \times a[2]  +a[3]/1.2}$

\vspace{0.5cm}
%\hrulefill

\begin{lstlisting}[flexiblecolumns=true,basicstyle=\sffamily]     
      block<double,5,5,AoSoA/AoS> block;
      typename block<double,5,5,AoSoA/AoS>::iterator it;
      for(it = block.begin();it != block.end(); ++it)
          (*it)[0] = (*it)[1]x(*it)[2]+(*it)[3]/1.2;
\end{lstlisting}
\vspace{0.5cm}

Simple SIMD wrappers are inefficient (too ASM/copy)

Solution : expression-template (Vandevoorde-2002)




\end{frame}



%%%%%%%%%%%%%%%%%%%%%%%%%% EXP 6

\subsection*{Template-expression: expression-template 2 }
\begin{frame}[fragile]
\centering
Associate the container with "expression-template"  method and a vector (SIMD) class
\begin{lstlisting}[flexiblecolumns=true,basicstyle=\sffamily]     
          (*it)[0] = (*it)[1]x(*it)[2]+(*it)[3]/1.2;
\end{lstlisting}
\begin{minipage}{0.4\textwidth}
\begin{tikzpicture}[auto,node distance=3 cm, scale = 1, transform shape,scale=0.5]
\node[initial,state] (A)                                        {Container};
\node[state]         (B) [right of=A]                       {Vector};
\node[state]         (C) [above of=B]                  {Math};
\node[state]         (D) [below of=B]                   {expression-template};

\path[<->] (A) edge [left]      node [align=center]  {AoS} (C)
               (A) edge [left]          node [align=center]  {AoSoA} (D)
               (B) edge [right]       node [align=center]  {} (D)
               (B) edge [right]       node [align=center]  {} (C);
\end{tikzpicture}
\end{minipage}
\begin{minipage}{0.4\textwidth}
\begin{tikzpicture}[every node/.style={circle,draw,scale=0.55},scale=0.7]
     \node[circle, draw,fill={C0}] (a){$=$} 
        child{node (b){$(*it)[0]$}}
        child{node[fill={C1}] (c){$+$}
            child { node[left=1.4cm,fill={C1}]  (d){$\times$}
                child {	node (e){$(*it)[1]$}}
                child {	node (f){$(*it)[2]$}}
            }
            child { node[circle,draw,fill={C1}]  (g){/}
                child { node (h) {$(*it)[3]$} }			
                child {	node (i) {1.2} }			
            }
        };
\end{tikzpicture}
\end{minipage}

\begin{center}
Reorder operations (recursive tree - compile time) \underline{without useless reg. mov}
\end{center}
\end{frame}



%%%%%%%%%%%%%%%%%%%%%%%%%% EXP 7

\subsection*{Template-expression: expression-template 3 }
\begin{frame}[fragile]
\begin{center}
 $\boxed{a[0] = a[1]  \times e^{a[2]}  +a[3]/1.2}$,  $ e^x = \sum_i^n a_ix^i$ (Remez)
\end{center}
\begin{minipage}{0.45\textwidth}
\begin{tikzpicture}[every node/.style={circle,draw,scale=0.4},scale=0.4]
     \node[circle, draw,fill={C0}] {$=$} 
        child{node {ymm0}}
        child{node [fill={C1}]  {$+$}
            child { node[left=1.1cm,fill={C1}]  {$\times$}
                child {	node {ymm1}}
                child {	node [fill={C1}] {+}
                   child {node {$a_0$}}
                   child {node [fill={C1}] {$\times$}
                       child {node {ymm2}}
                       child {node [fill={C1}] {$+$}
                          child {node {$a_1$}}
                          child {node [fill={C1}] {$\times$}
                             child {node {ymm2}}
                             child {node [fill={C1}] {$+$}
                                 child {node {$a_2$}}
                                 child {node [fill={C1}] {$\times$}
                                     child {node {ymm2}}
                                     child {node {$a_3$}}
                                 }
                             }
                          }
                       }
                   }
                }
            }
            child { node[circle,draw, fill={C1}]  {/}
                child { node  {ymm3} }			
                child {	node  {1.2} }			
            }
        };
\end{tikzpicture}
\end{minipage}
\begin{minipage}{0.4\textwidth}
\begin{tikzpicture}[every node/.style={circle,draw,scale=0.4},scale=0.4]
     \node[circle, draw,fill={C0}] {$=$} 
        child{node {ymm0}}
        child{node [fill={C1}]  {fma}
                child {	node {ymm1}}
                child {	node [fill={C1}] {fma}
                   child {node {ymm2}}
                   child {node {$a_0$}}
                       child {node [fill={C1}] {fma}
                       child {node {ymm2}}
                       child {node {$a_1$}}
                           child {node [fill={C1}] {fma}
                           child {node {ymm2}}
                           child {node {$a_2$}}
                                child {node  {$a_3$}
                          }
                       }
                   }
                }
            child { node[circle,draw, fill={C1}, right=2cm]  {/}
                child { node  {ymm3} }			
                child {	node  {1.2} }			
            }
        };
\end{tikzpicture} \\
FMA support (not fully)
\begin{itemize} 
\item BGQ 
\item Haswell (Intel proc.)
\item MIC (Intel acc.)
\end{itemize}
\end{minipage}

\end{frame}

%%%%%%%%%%%%%%%%%%%%%%%%%% EXP 8
\section[Benchmark]{Benchmarks}
\subsection*{Benchmarks}
\begin{frame}[fragile]
\centering
Two benchmarks:
\begin{eqnarray}
a[0] &=& a[4] \times a[5] + a[1] \times e^{a[2]} \\
a[0] &=& a[5]/(a[1] \times a[2]+a[3]-a[4])
\end{eqnarray}


\begin{itemize}
\item X86 : GCC 4.8, Clang 3.2, Intel 13.0 
\item BGQ: Clang 3.2, XLC 12.1
\item MIC: Intel 13.2
\item Option of compilation X86: -O3 -avx or -sse2 
\item Option of compilation PPC:  -O3 -qhot
\item \verb+block<float/double,8,1024, AoS/AoSoA> block+
\item Unit: [cycle] , \verb+rdtsc+ counter
\end{itemize}

\end{frame}


%%%%%%%%%%%%%%%%%%%%%%%%%% EXP 8 bis float
\subsection*{$1^{st}$ Benchmark, $a[0] = a[4] \times a[5] + a[1] \times e^{a[2]}$}
\begin{frame}[fragile]
\begin{center}
$a[0] = a[4] \times a[5] + a[1] \times e^{a[2]}$, X86 cluster
\end{center}
\begin{tabular}{ c |  c c | c c }
\color{C0}{float}       & \multicolumn{2}{c |}{AoS} & \multicolumn{2}{c}{AoSoA}\\
                        & \multicolumn{2}{c | }{scalar} & AVX 256b & SSE 128b\\
                        \hline
   Clang - X86 & 115 007 & 124 404  & 22 747$^2$ & \cellcolor{C2}13 149$^2$\\
   GCC - X86  & 113 631 & 115 220  & \cellcolor{C2}9 627$^2$ & 15 555$^2$ \\
   Intel - X86   & 19 323$^1$  & 17 606$^1$ & \cellcolor{C2}8 293$^3$ & 11 181$^3$\\
   \hline
\end{tabular}\\
\vspace{0.5cm}
$^1$ Intel math, $^2$ my vec exp,  $^3$ Intel SVML, no FMA, $e$ performance decisive 
\end{frame}


%%%%%%%%%%%%%%%%%%%%%%%%%% EXP 8 bis double
\subsection*{$1^{st}$ Benchmark, $a[0] = a[4] \times a[5] + a[1] \times e^{a[2]}$}
\begin{frame}[fragile]
\begin{center}
$a[0] = a[4] \times a[5] + a[1] \times e^{a[2]}$, X86 cluster
\end{center}
\begin{tabular}{ c |  c c | c c }
\color{C0}{double}           & \multicolumn{2}{c |}{AoS} & \multicolumn{2}{c}{AoSoA}\\
                        & \multicolumn{2}{c | }{scalar} & AVX 256b & SSE 128b\\
   \hline
   Clang - X86 & 112 263 & 133 821 & 41 059$^2$ & \cellcolor{C2}26 247$^2$\\
   GCC - X86  & 110 554 & 144 569  & \cellcolor{C2}17 891$^2$ & 27 290$^2$ \\
   Intel - X86   & 37 102$^1$ & 28 226$^1$  & 19 223$^3$ & \cellcolor{C2}16 105$^3$\\
   \hline
\end{tabular}\\
\vspace{0.5cm}
$^1$ Intel math, $^2$ my vec exp, $^3$ Intel SVML, no FMA
\end{frame}


%%%%%%%%%%%%%%%%%%%%%%%%%% EXP 8 bis double double
\subsection*{$1^{st}$ Benchmark, $a[0] = a[4] \times a[5] + a[1] \times e^{a[2]}$}
\begin{frame}[fragile]
\begin{center}
$a[0] = a[4] \times a[5] + a[1] \times e^{a[2]}$, BGQ
\vspace{0.5cm}

\begin{tabular}{ c |  c c | c c }
\color{C0}{double}           & \multicolumn{2}{c |}{AoS} & \multicolumn{2}{c}{AoSoA}\\
                        & \multicolumn{2}{c |}{scalar} & \multicolumn{2}{c }{QPX 256b} \\
   \hline
   XLC - BGQ  & \multicolumn{2}{c |}{  129 591$^1$}&  \multicolumn{2}{c |}{\cellcolor{C2}$48\, 666^2$} \\
   XLC - BGQ  & \multicolumn{2}{c |}{  129 591$^1$}&  \multicolumn{2}{c |}{\cellcolor{C2}$72\, 468^3$} \\
   Clang - BGQ & \multicolumn{2}{c |}{127 784$^1$} & \multicolumn{2}{c |}{\cellcolor{C2}$23\, 830^2$}\\
   Clang - BGQ & \multicolumn{2}{c |}{127 784$^1$} & \multicolumn{2}{c |}{\cellcolor{C2}$54\, 071^3$}\\
   \hline
\end{tabular}\\
\end{center}

$^1$ IBM mass, $^2$ my vec exp, $^3$ IBM simd\_mass,  FMA

\vspace{0.5cm}
\end{frame}


%%%%%%%%%%%%%%%%%%%%%%%%%% EXP 8
%\subsection*{$1^{st}$ Benchmark}
%\begin{frame}[fragile]
%\centering
%
%Evaluate $e$ from vendor (my precision $10^{-8}$ on [-10:10])
%
%\vspace{1cm}
%$\boxed{a[0] = a[4] \times a[5] + a[1] \times e^{a[2]}}$ \\
%\vspace{1cm}
%
%
%\begin{tabular}{ c c c  }
%     \verb+float+                    & AoSoA/AVX & AoSoA/SSE \\
%                          \hline
%   GCC - X86  & 9 627 & 15 555 \\
%   Intel - $e_{v}$ - X86   & \cellcolor{C2}8 293 & \cellcolor{C2}11 181 \\
%   \hline
%        \verb+double+                    &  &  \\
%   GCC - X86  &    \cellcolor{C2}17 891 & 27 290 \\
%   Intel - $e_{v}$ - X86   &   19 223 & \cellcolor{C2}16 105\\
%   XLC - BGQ                  &   271 327 &  - \\
%   XLC - $e_{v}$ - BGQ   &  134 154 & - \\
%\end{tabular}
%
%
%\end{frame}

%%%%%%%%%%%%%%%%%%%%%%%%%% EXP 9

\subsection*{$2^{nd}$ Benchmark, $a[0] = a[5]/(a[1] \times a[2]+a[3]-a[4])$}
\begin{frame}[fragile]
\centering
$\boxed{a[0] = a[5]/(a[1] \times a[2]+a[3]-a[4])}$

\begin{tabular}{ c c c | c c }
\color{C0}{float}           & \multicolumn{2}{c |}{AoS} & \multicolumn{2}{c}{AoSoA}\\
                        & \multicolumn{2}{c |}{scalar}& AVX 256b & SSE 128b\\

                          \hline
   GCC - X86  & 11 366 & 11 357 & 2 912 & \cellcolor{C2}2 894 \\
   Clang- X86  & 11 686 & 11 488 & 2 990 & \cellcolor{C2}2 903 \\
   Intel  - X86   & 11 446 & 11 345 & 2 924 &\cellcolor{C2} 2 679\\
      \hline
        \color{C0}{double}                    &  & & & \\
   GCC - X86  & 17 817 & 17 808  & 9 296 & \cellcolor{C2}8 932 \\
   Clang- X86  & 17 814 & 17 799 & 8 950 & \cellcolor{C2}7 535 \\
   Intel  - X86   & 17 813 & 17 796 & 9 006 & \cellcolor{C2}8 944\\
   \hline
 \color{C0}{double}                         & QPX 256b &  & QPX 256b & \\
   XLC - BGQ   &  \cellcolor{C2}30 211 &        -   & 52 177 &  - \\
   Clang - BGQ   &  77 344 &        -   & \cellcolor{C2}49 022 &  - \\
\hline

\end{tabular}

\vspace{0.5cm}

note: vec\_div is scalar on BGQ, break SIMD optimization

\end{frame}

%%%%%%%%%%%%%%%%%%%%%%%%%% EXP 9

\subsection*{$a[0] = a[4] \times a[5] + a[1] \times e^{a[2]}$ and $a[0] = a[5]/(a[1] \times a[2]+a[3]-a[4])$ on MIC}
\begin{frame}[fragile]
\centering
MIC benchmark on one core:

\begin{tabular}{ c c  |  c }
      & AoS & AoSoA\\
\color{C0}{float}                                     & scalar & 512b SIMD reg. \\
                           \hline
   Bench 1  & 78 225 & \cellcolor{C2}15 945 \\
   Bench 2  & 68 640 & \cellcolor{C2}12 862  \\
      \hline
        \color{C0}{double}                    &  &  \\
        Bench 1  & 114 629 & \cellcolor{C2}49 642 \\
   Bench 2  & 99 126 &\cellcolor{C2} 29 961  \\
\hline
\end{tabular}
\vspace{0.5cm}

Note: 512 bits reg. are $2 \times 256$ bits reg., Intel SVML



\end{frame}

%%%%%%%%%%%%%%%%%%%%%%%%%% EXP 10

%\subsection*{$3^{rd}$ Benchmark}
%\begin{frame}[fragile]
%\begin{eqnarray}
%          a[0] &=& a[1] \times a[2]+a[6] + 3.14 \nonumber \\
%          a[1] &=& a[2] \times a[3]+a[6] + 2.18 \nonumber \\
%          a[2] &=& a[3] \times a[4]+a[6] + 4.25 \nonumber \\
%          a[3] &=& a[4] \times a[5]+a[6] + 3 \nonumber 
%\end{eqnarray}
%
%\begin{tabular}{ c c c c c }
%   \verb+float+                  & AoS/AVX & AoS/SSE & AoSoA/AVX & AoSoA/SSE \\
%                          \hline
%   Clang - X86 & 10 985 & 11 033 & 3 057 &\cellcolor{C2}2 547\\
%   GCC - X86  & 12 239 & 12 305 & \cellcolor{C2}2 529 & 2 534 \\
%   Intel - X86   & 10 014 & 9 857 & \cellcolor{C2}7 776 & 10 522\\
%         \hline
%        \verb+double+                    &  & & & \\
%   Clang - X86 &  12 784 & 7 940 & 5 961 & \cellcolor{C2}5 090\\
%   GCC - X86  & 12 278 & 8 525  & 5 033 & \cellcolor{C2}4 968 \\
%   Intel - X86   & 12 158 & \cellcolor{C2}9 351 & 15 398 & 20 913\\
%   XLC - BGQ & \cellcolor{C2} 68 245 & - & 163 993 & - \\
%\end{tabular}
%
%\end{frame}

%%%%%%%%%%%%%%%%%%%%%%%%%% EXP 11

\subsection*{Conclusions}
\begin{frame}[fragile]
Conclusions:
\begin{itemize}
\item Very promising, whatever the expression
\item Framework works on 3 different hardware (easy to extend one day per arch.)
\item AVX not faster than SSE, I get some penalty in my ASM code, I do not why (out of my expertise, order of execution, ASM penalty)
\item Intel math lib: excellent
\item Clang good surprise
\item regression tests with a lot of torture tests, validation ok
\item Finest analysis requires read the processor manual (very long)
\end{itemize}

\end{frame}


\end{document}

